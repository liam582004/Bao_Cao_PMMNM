Mục tiêu chính của Spotify là "democratize audio" – dân chủ hóa việc tiếp cận âm thanh. Điều này có nghĩa là cung cấp cho người dùng trên toàn thế giới quyền truy cập dễ dàng và hợp pháp vào kho nội dung âm thanh phong phú, đồng thời tạo cơ hội cho các nghệ sĩ, dù lớn hay nhỏ, tiếp cận với khán giả toàn cầu mà không cần thông qua các kênh phân phối truyền thống. Spotify cũng đặt mục tiêu không ngừng cải thiện trải nghiệm người dùng thông qua việc ứng dụng công nghệ tiên tiến như trí tuệ nhân tạo để cá nhân hóa nội dung và đề xuất âm nhạc phù hợp với sở thích của từng cá nhân. 

Ngoài ra, Spotify còn hướng đến việc mở rộng hệ sinh thái âm thanh của mình bằng cách tích hợp podcast và audiobook, biến nền tảng này thành điểm đến toàn diện cho mọi nhu cầu nghe của người dùng. Điều này không chỉ tăng cường giá trị cho người dùng mà còn tạo thêm nguồn thu nhập cho các nhà sáng tạo nội dung. 

Tóm lại, Spotify được phát triển với mục tiêu cung cấp một giải pháp nghe nhạc trực tuyến hợp pháp, chất lượng cao và thuận tiện, đồng thời hỗ trợ các nghệ sĩ tiếp cận khán giả rộng rãi hơn. Nền tảng này không ngừng đổi mới và mở rộng để đáp ứng nhu cầu ngày càng cao của người dùng và thị trường âm nhạc toàn cầu.