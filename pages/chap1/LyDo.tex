Trước khi Spotify xuất hiện, ngành công nghiệp âm nhạc đối mặt với nhiều thách thức, đặc biệt là vấn đề vi phạm bản quyền do việc chia sẻ nhạc trái phép trên các nền tảng như Napster. Daniel Ek và Martin Lorentzon, những người sáng lập Spotify, nhận thấy cần thiết phải tạo ra một dịch vụ âm nhạc trực tuyến hợp pháp, cung cấp trải nghiệm nghe nhạc chất lượng cao và thuận tiện, đồng thời đảm bảo quyền lợi cho các nghệ sĩ và nhà sản xuất. Mục tiêu của họ là cung cấp một giải pháp thay thế hấp dẫn hơn so với việc tải nhạc bất hợp pháp, bằng cách mang đến cho người dùng quyền truy cập tức thì vào một thư viện âm nhạc khổng lồ với chất lượng cao. 