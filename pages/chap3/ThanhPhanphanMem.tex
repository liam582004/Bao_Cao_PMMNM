\subsubsection{ Frontend:}
\begin{itemize}
    \item \textbf{ReactJS:} React là một thư viện JavaScript mã nguồn mở, được phát triển bởi Facebook vào năm 2013, nhằm xây dựng giao diện người dùng cho các ứng dụng web. React cho phép các lập trình viên phát triển các thành phần giao diện (components) một cách hiệu quả, dễ bảo trì và tái sử dụng. Với React, người dùng có thể trải nghiệm giao diện mượt mà và dễ dàng tương tác với ứng dụng.
\end{itemize}

\subsubsection{ Backend:}
\begin{itemize}
    \item \textbf{Python - Django:} Django là một framework Python mạnh mẽ giúp phát triển các ứng dụng web nhanh chóng và dễ dàng hơn. Django cung cấp các tính năng như ORM (Object Relational Mapping) để quản lý cơ sở dữ liệu, giúp việc phát triển các API RESTful trở nên đơn giản và thuận tiện. Nó cũng hỗ trợ bảo mật và quản lý người dùng, giúp dễ dàng xử lý các yêu cầu API của người dùng.
\end{itemize}

\subsection{Hệ quản trị cơ sở dữ liệu:}
\begin{itemize}
    \item \textbf{MySQL:} MySQL là một hệ quản trị cơ sở dữ liệu mã nguồn mở, sử dụng SQL để truy vấn và thao tác với dữ liệu. Đây là lựa chọn phổ biến cho các ứng dụng web vì tính ổn định và khả năng tương thích với nhiều hệ điều hành khác nhau. Trong ứng dụng Clone Spotify, MySQL sẽ lưu trữ dữ liệu người dùng, danh sách bài hát, album, playlist, cũng như các dữ liệu liên quan đến người dùng và nhạc.
\end{itemize}

\subsection{Giao thức truyền thông:}
\begin{itemize}
    \item \textbf{WebSocket:} WebSocket là một giao thức truyền thông giúp thiết lập kênh truyền thông hai chiều giữa máy chủ và máy khách. WebSocket cung cấp kết nối liên tục, giúp việc phát trực tiếp và truyền tải dữ liệu âm nhạc giữa máy chủ và người dùng hiệu quả, tối ưu hóa trải nghiệm nghe nhạc và giảm độ trễ khi phát nhạc.
\end{itemize}

