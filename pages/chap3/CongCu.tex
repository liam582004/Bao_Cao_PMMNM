\subsection{IntelliJ IDEA:}
\begin{itemize}
    \item IntelliJ IDEA là IDE đầu tiên của JetBrains. Nó chủ yếu nhắm vào các công nghệ dựa trên JVM như Java, Groovy, Kotlin và Scala nhưng nó cũng hỗ trợ Java EE và phát triển web. Phiên bản đầu tiên của IntelliJ IDEA được phát hành vào tháng 1 năm 2001. Ở đây công cụ này được dùng để code spring boot API.
\end{itemize}

\subsection{Visual Studio Code (VS Code):}
\begin{itemize}
    \item Visual Studio Code là một trình soạn thảo mã nguồn nhẹ nhưng mạnh mẽ, hỗ trợ đa nền tảng như Windows, macOS và Linux. Với các tính năng như hoàn thành mã thông minh, tích hợp Git và hỗ trợ nhiều tiện ích mở rộng, VS Code được sử dụng để phát triển frontend với ReactJS trong dự án Clone Spotify.
\end{itemize}

\subsection{Postman:}
\begin{itemize}
    \item Postman là một trong những công cụ phổ biến nhất được sử dụng trong thử nghiệm các API. Như ta đã biết, API chịu trách nhiệm kết nối các ứng dụng với nhau, có Postman sẽ giúp cho thao tác với API này trở nên dễ dàng hơn. Thông thường, Postman sẽ được dùng cho API kiểu REST. Với Postman, ta có thể gọi Rest API mà không cần viết dòng code nào. Postman hỗ trợ tất cả các phương thức HTTP (GET, POST, PUT, PATCH, DELETE, …). Bên cạnh đó, nó còn cho phép lưu lại lịch sử các lần request, rất tiện cho việc sử dụng lại khi cần.
\end{itemize}

\subsection{MySQL Workbench:}
\begin{itemize}
    \item MySQL Workbench là công cụ giúp quản lý cơ sở dữ liệu MySQL, cho phép phát triển mô hình dữ liệu và quản lý cơ sở dữ liệu MySQL. Nó cung cấp giao diện đồ họa cho phép dễ dàng tạo, chỉnh sửa cơ sở dữ liệu và thực hiện các thao tác như đảo ngược (reverse engineering) và chuyển tiếp (forward engineering) cơ sở dữ liệu.
\end{itemize}

\subsection{GitHub:}
\begin{itemize}
    \item GitHub là công cụ quản lý mã nguồn phổ biến, cho phép các lập trình viên chia sẻ, cộng tác và quản lý phiên bản mã nguồn. Sự phát triển của nền tảng GitHub bắt đầu vào ngày 19 tháng 10 năm 2007. Trang web được đưa ra vào tháng 4 năm 2008 do Tom Preston-Werner, Chris Wanstrath, và PJ Hyett .Microsoft đã mua GitHub vào tháng 6 năm 2018.
\end{itemize}

\subsection{ Deploy: Amazon EC2 và VPS:}
\begin{itemize}
    \item \textbf{Amazon EC2:} Amazon EC2 là dịch vụ đám mây của Amazon Web Services (AWS), cho phép người dùng thuê máy chủ ảo (instances) để triển khai và chạy ứng dụng. EC2 mang lại sự linh hoạt, khả năng mở rộng và tính sẵn sàng cao cho các ứng dụng web. Với Clone Spotify, EC2 sẽ được sử dụng để triển khai phần backend và giúp ứng dụng có thể mở rộng khi lượng người dùng tăng.
    \item \textbf{VPS:} VPS là một máy chủ ảo được phân chia từ một máy chủ vật lý duy nhất. Mỗi VPS có hệ điều hành riêng biệt, giúp triển khai ứng dụng và quản lý tài nguyên như một máy chủ độc lập. Với Clone Spotify, VPS sẽ được sử dụng cho việc triển khai các dịch vụ khác ngoài backend, như các ứng dụng phụ trợ hoặc lưu trữ dữ liệu không đụng đến hệ thống chính.
\end{itemize}