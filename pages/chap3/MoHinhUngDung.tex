\noindent{\normalsize \textbf{Frontend:}}
\begin{itemize}
    \item \textbf{ReactJS:} React là một thư viện JavaScript mã nguồn mở, được phát triển bởi Facebook vào năm 2013, nhằm xây dựng giao diện người dùng cho các ứng dụng web. React cho phép các lập trình viên phát triển các thành phần giao diện (components) một cách hiệu quả, dễ bảo trì và tái sử dụng. Với React, người dùng có thể trải nghiệm giao diện mượt mà và dễ dàng tương tác với ứng dụng.
\end{itemize}

\noindent{\normalsize \textbf{Backend:}}
\begin{itemize}
    \item \textbf{Python - Django:} Django là một framework Python mạnh mẽ giúp phát triển các ứng dụng web nhanh chóng và dễ dàng hơn. Django cung cấp các tính năng như ORM (Object Relational Mapping) để quản lý cơ sở dữ liệu, giúp việc phát triển các API RESTful trở nên đơn giản và thuận tiện. Nó cũng hỗ trợ bảo mật và quản lý người dùng, giúp dễ dàng xử lý các yêu cầu API của người dùng.
\end{itemize}

\noindent{\normalsize \textbf{Hệ quản trị cơ sở dữ liệu:}}
\begin{itemize}
    \item \textbf{MySQL:} là một hệ thống quản trị cơ sở dữ liệu mã nguồn mở (Relational Database Management System, viết tắt là RDBMS), thuộc quyền sở hữu của Oracle, được sử dụng để quản lý và lưu trữ dữ liệu. Nó sử dụng SQL (Structured Query Language) làm ngôn ngữ chính để truy vấn và thao tác với cơ sở dữ liệu. MySQL phổ biến trong các ứng dụng web, đặc biệt là các ứng dụng sử dụng kiến trúc LAMP (Linux, Apache, MySQL, PHP/Python/Perl).Các ứng dụng web lớn nhất như Facebook, Twitter, YouTube, Google, và Yahoo! đều dùng MySQL cho mục đích lưu trữ dữ liệu. Nó đã tương thích với nhiều hạ tầng máy tính quan trọng như Linux, macOS, Microsoft Windows, và Ubuntu.
\end{itemize}

\noindent{\normalsize \textbf{Giao thức truyền thông:}}
\begin{itemize}
    \item \textbf{WebSocket:} là một giao thức truyền thông giúp cho việc thiết lập kênh truyền thông hai chiều giữa máy chủ và máy khách. WebSocket hoạt động bằng cách thiết lập kết nối HTTP liên tục với máy chủ và sau đó nâng cấp nó lên kết nối websocket hai chiều bằng cách gửi Upgrade header. WebSocket được hỗ trợ trong hầu hết các trình duyệt web hiện đại và cho các trình duyệt không hỗ trợ, chúng tôi có các thư viện cung cấp dự phòng cho các kỹ thuật khác như Comet và HTTP Long Polling.
\end{itemize}





